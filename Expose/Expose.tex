\documentclass[german,a4paper,12pt,parskip=half]{scrartcl}
\usepackage[T1]{fontenc} % utf8 <- produce real utf8 characters
\usepackage[utf8]{inputenc} % utf8 <- accept utf8 input characters
\usepackage{datetime}
\usepackage[german]{babel}
\usepackage[automark,headsepline]{scrlayer-scrpage}
\usepackage[hscale=0.75,vscale=0.75,vmarginratio={85:100},heightrounded]{geometry} % less margin at bottom
\usepackage[svgnames]{xcolor}
\usepackage{hyperref}
\usepackage[clean]{svg}
\usepackage{float}
\usepackage{minted}
\usepackage{graphicx} % for pdf and other graphics
\usepackage{cite} % cite
\usepackage{anyfontsize} % mute warnings?
\usepackage{microtype} % Subliminal refinements towards typographical perfection - eg. Hypernation
\usepackage{xspace} % set a space if not fullstop / end of sentence
\usepackage{calc} % for lengh calculation
\usepackage{enumitem}
\usepackage{tabularx}

%---------------------------------
% Shut up I know
%---------------------------------
\setlength{\belowcaptionskip}{-1em}

%---------------------------------
% Shut up I know
%---------------------------------
\pdfsuppresswarningpagegroup=1

%---------------------------------
% Cite Style
%---------------------------------
\bibliographystyle{plain}

%---------------------------------
% Make the Part Heading a little bit smaller
%---------------------------------
\setkomafont{part}{\LARGE}

%---------------------------------
% SVG image configuration
%---------------------------------
\setsvg{inkscape = inkscape -z -C} % better svgs

%---------------------------------
% No Page break
%---------------------------------
\newenvironment{absolutelynopagebreak}
  {\par\nobreak\vfil\penalty20\vfilneg
   \vtop\bgroup}
  {\par\xdef\tpd{\the\prevdepth}\egroup
   \prevdepth=\tpd}

%---------------------------------
% RoadMap
%---------------------------------
\newlength{\myRMLength}
\newcommand\myRoadMapTime[2]{%
    \ifthenelse{\equal{#2}{}}%
    {% if
      #1%
    }{% else
      \setlength{\myRMLength}{0.63\textwidth-(\textwidth-\linewidth)}%
      \noindent\hbox to \myRMLength{#1\hfill}(#2)%
    }%
}%

%---------------------------------
% Signatures
%---------------------------------
\makeatletter
\newcommand{\mySignatures}{% We don't consume the args so they stay for the next
  \@ifnextchar\bgroup{% if we have arguments
    \mySigTableHead%
    \mySigGenerateBody%
  }{% else not a single argument
    \errmessage{The macro needs at least one argument}%
  }%
}%
\newcommand{\mySigGenerateBody}[1]{%
  \mySigTableBody{#1}% Make current argument
  \@ifnextchar\bgroup{% if another argument is present
    \mySigGenerateBody% apply it
  }{% else finish the table and figure
    \mySigTableBottom%
  }%
}% Gobble next "argument"
\newcommand{\mySigTableHead}{%
  % begin the table
  \begin{figure}[H]%
  \centering%
  \begin{tabular}{ccc}%
}
\newcommand{\mySigTableBody}[1]{%
  % Apply current argument
  {}%
    &{}%
      &{}\\%
  \makebox[4cm]{\hrulefill}%
    & \makebox[3cm]{\hrulefill}%
      & \makebox[6cm]{\hrulefill}\\%
  Ort%
    & Datum%
      & #1\\[1.5em]%
}
\newcommand{\mySigTableBottom}{%
  % end the table
  \end{tabular}%
  \end{figure}%
}
\makeatother

%---------------------------------
% Minted Code
%---------------------------------
\newcommand\myCMinco[5][cpp]{
  {
    \ifthenelse{\equal{#2}{}}
    {}
    {\noindent\texttt{#2}\\[-2em]}%
    \inputminted[
        frame=lines,
        framesep=2mm,
        bgcolor=LightGray!20!White,
        fontsize=\footnotesize,
        linenos,
        firstline=#4,
        lastline=#5,
      ]{#1}{#3}
  }
}
\newcommand\myMinco[5][cpp]{
  {
    \ifthenelse{\equal{#2}{}}
    {}
    {\noindent\texttt{#2}\\[-2em]}%
    \inputminted[
        frame=lines,
        framesep=2mm,
        fontsize=\footnotesize,
        linenos,
        firstline=#4,
        lastline=#5,
      ]{#1}{#3}
  }
}

%---------------------------------
% Minted inline
%---------------------------------
\newcommand{\myMinin}[1]{\mintinline{c++}{#1}}

%---------------------------------
% Meta Variables
%---------------------------------
\newcommand{\MetaInstitute}{Hochschule Bremen}
\newcommand{\MetaUnit}{Bachelorarbeit}
\newcommand{\MetaTask}{Expos\'e}
\newcommand{\MetaTitle}{Automatisierung durch Hilfe von Macros}
\newcommand{\MetaAuthorName}{Roland}
\newcommand{\MetaAuthorSurname}{Jäger}
\newcommand{\MetaStudentNumber}{360956}
\newcommand{\MetaAuthor}{\MetaAuthorName~\MetaAuthorSurname}

%---------------------------------
% Horizontal line for title page
%---------------------------------
\newcommand{\HRule}{\rule{\linewidth}{0.2mm}}

%---------------------------------
% href Link as foot note
%---------------------------------
\newcommand\myFnurl[2]{
  \href{#2}{#1}\footnote{\url{#2}}
}

%---------------------------------
% Meta data and Link Colour
%---------------------------------
\newcommand\myShade{70}

\definecolor{mylinkcolor}{RGB}{113, 31, 155}
\definecolor{mycitecolor}{RGB}{255, 189, 76}
\definecolor{myurlcolor}{RGB}{62, 106, 171}

\hypersetup{
  pdfauthor   = {\MetaAuthor},
  pdftitle    = {\MetaTitle},
  pdfsubject  = {\MetaUnit, \MetaTask},
  % pdfsubject  = {\MetaUnit},
  % pdfkeywords = {\MetaTitle, \MetaUnit, \MetaInstitute},
  pdfkeywords = {\MetaTitle, \MetaUnit, \MetaTask, \MetaInstitute},
  colorlinks  = true,
  linkcolor   = mylinkcolor!\myShade!black,
  citecolor   = mycitecolor!\myShade!black,
  urlcolor    = myurlcolor!\myShade!black,
}

%---------------------------------
% Minted color scheme
%---------------------------------
\usemintedstyle{borland}

%---------------------------------
% Fancy header
%---------------------------------
\clearpairofpagestyles
\lohead{\headmark}
\cofoot[\pagemark]{\pagemark}
\pagestyle{scrheadings}

%---------------------------------
% Start page count
%---------------------------------
\setcounter{page}{0}

%---------------------------------
% Document start
%---------------------------------
\begin{document}
%---------------------------------
% Titlepage
%---------------------------------
\begin{titlepage}
  \shortdate % Use Short Date
  \center % Center everything on the page

  %---------------------------------
  % HEADING SECTIONS
  %---------------------------------
  ~\\[2cm]
  % ~\\[2.5cm]
  \textsc{\LARGE \MetaInstitute}\\[1.5cm] % Name of your university/college
  \textsc{\Large \MetaUnit}\\[0.5cm] % Major heading such as course name
  \textsc{\large \MetaTask}\\[2cm] % Minor heading such as course title

  %---------------------------------
  % TITLE SECTION
  %---------------------------------
  \HRule \\[0.5cm]
  {
    \LARGE \bfseries \MetaTitle \\[0.50cm] % Title of your document
    \par
  }
  \HRule \\[1.5cm]

  %---------------------------------
  % AUTHOR SECTION
  %---------------------------------
  \begin{minipage}{0.4\textwidth}
    \begin{flushleft}
      \large
        \emph{Author:}\\
        \MetaAuthorName~\textsc{\MetaAuthorSurname}\\
      \small \MetaStudentNumber
    \end{flushleft}
  \end{minipage}
  ~
  \begin{minipage}{0.4\textwidth}
    \begin{flushright}

    \end{flushright}
  \end{minipage}\\[4cm]

  %---------------------------------
  % DATE SECTION
  %---------------------------------
  \vspace*{\fill}
  {
    \large \today
  }
\end{titlepage}


%---------------------------------
% table of contents
%---------------------------------
\tableofcontents
\newpage

%---------------------------------
% Einleitung
%---------------------------------
\section{Einleitung}
\label{sec:einleitung}

%---------------------------------
% Problemstellung
%---------------------------------
\section{Problemstellung}
\label{sec:problemstellung}

%---------------------------------
% Lösungsansatz
%---------------------------------
\section{Lösungsansatz}
\label{sec:loesungsansatz}

%---------------------------------
% Konkrete Aufgaben
%---------------------------------
\section{Konkrete Aufgaben}
\label{sec:konkrete_aufgaben}

%---------------------------------
% Arbeitsumfeld
%---------------------------------
\section{Arbeitsumfeld}
\label{sec:arbeitsumfeld}

  %---------------------------------
  % Literatur
  %---------------------------------
  \subsection{Literatur}
  \label{ssec:literatur}
  \nocite{Johnson:1997:FRA:262793.262799}
  \bibliography{lit}

  %---------------------------------
  % Software
  %---------------------------------
  \subsection{Software}
  \label{ssec:software}

  %---------------------------------
  % Hardware
  %---------------------------------
  \subsection{Hardware}
  \label{ssec:hardware}


%---------------------------------
% Planung
%---------------------------------
\section{Planung}
\label{sec:planung}

  %---------------------------------
  % Wann
  %---------------------------------
  \subsection{Wann}
  \label{ssec:wann}
    März bis Juni 2015

  %---------------------------------
  % Wo
  %---------------------------------
  \subsection{Wo}
  \label{ssec:wo}
    \begin{tabular}{l}
      P3 engineering GmbH\\
      Flughafenallee 26/28\\
      28199 Bremen\\
      \href{www.p3-group.com}{www.p3-group.com}\\
    \end{tabular}

  %---------------------------------
  % Arbeitspakete
  %---------------------------------
  \subsection{Arbeitspakete}
  \label{ssec:arbeitspakete}
  \begin{itemize}%
    \item \myRoadMapTime{Recherche}{ca.1 Woche}%
    \item \myRoadMapTime{Konzeption}{ca.1 Woche}%
      \begin{itemize}%
        \item \myRoadMapTime{Serverarchitektur / Kontrollmechanismen}{}%
        \item \myRoadMapTime{Mobiler Client}{}%
        \item \myRoadMapTime{Webclient}{}%
      \end{itemize}%
    \item \myRoadMapTime{Implementierung inkl. Dokumentation}{ca.1 Woche}%
      \begin{itemize}%
        \item \myRoadMapTime{Serverarchitektur / Kontrollmechanismen}{ca.1 Woche}%
        \item \myRoadMapTime{Mobiler Client}{ca.1 Woche}%
        \item \myRoadMapTime{Webclient}{ca.1 Woche}%
      \end{itemize}%
    \item \myRoadMapTime{Testen}{ca.1 Woche}%
    \item \myRoadMapTime{Verfassen der BT}{ca.1 Woche}%
      \begin{itemize}%
        \item \myRoadMapTime{Einleitung + Anforderungsanalyse}{ca. 1 Woche, ab 10. Oktober}%
          \begin{itemize}%
            \item \myRoadMapTime{Kapitel 1: Allgemeines}{}%
            \item \myRoadMapTime{Kapitel 2: Einleitung}{}%
            \item \myRoadMapTime{Kapitel 3: Anforderungsanalyse}{}%
          \end{itemize}%
        \item \myRoadMapTime{Hauptteil}{ca. 2,5 Wochen, ab 17. Oktober}%
          \begin{itemize}%
            \item \myRoadMapTime{Kapitel 4: Grundlagen und alternative Lösungen}{}%
            \item \myRoadMapTime{Kapitel 5: Konzeption}{}%
            \item \myRoadMapTime{Kapitel 6: Exemplarische Realisation}{}%
          \end{itemize}%
        \item \myRoadMapTime{Schlussteil}{< 1 Woche, ab 7. November}%
          \begin{itemize}%
            \item \myRoadMapTime{Kapitel 7: Zusammenfassung und Ausblick}{}%
            \item \myRoadMapTime{Kapitel 8: Anhang}{}%
          \end{itemize}%
      \end{itemize}%
  \end{itemize}%

  %---------------------------------
  % Meilensteine
  %---------------------------------
  \subsection{Meilensteine}
  \label{ssec:meilensteine}

    \begin{table}[H]
      \begin{tabularx}{\textwidth}{lX}%
        \textbf{22. August:}        & Abschluss der Recherche / Beginn der Konzeption\\%
        \textbf{29. August:}        & Abschluss der Konzeption / Beginn der Implementierung\\%
        \textbf{10. Oktober:}        & Abschluss der Implementierung, Dokumentation und Testphase / Beginn der schriftlichen Arbeit\\%
        \textbf{17. Oktober:}        & Beginn der schriftlichen Arbeit am Hauptteil\\%
        \textbf{10. November:}        & Abschluss der schriftlichen Arbeit / Beginn der Korrektur, Binden der DA etc.\\%
        \textbf{15. November:}        & Abgabe der Arbeit\\%
      \end{tabularx}
    \end{table}

%---------------------------------
% Gliederung der Arbeit
%---------------------------------
\section{Gliederung der Arbeit}
\label{sec:gliederung_der_arbeit}
  \begin{enumerate}[
    labelindent=*,
    % style=multiline,
    leftmargin=\widthof{\textbf{Kapitel~0:~}},
    label=\arabic*.
  ]%
    \item[]%
      Allgemeines%
      \begin{enumerate}[label=\theenumi\arabic*.]%
          \item[]%
            Eidesstattliche Erklärung%
          \item[]%
            Danksagung%
      \end{enumerate}%
    \setcounter{enumi}{1}
    \item[\textbf{Kapitel~\arabic{enumi}:}]%
      Einleitung%
      \begin{enumerate}[label=\theenumi\arabic*.]%
          \item Problemfeld%
          \item Ziele der Arbeit%
          \item Hintergründe und Entstehung des Themas%
          \item Struktur der Arbeit, wesentliche Inhalte der Kapitel%
      \end{enumerate}%
    \setcounter{enumi}{2}
    \item[\textbf{Kapitel~\arabic{enumi}:}]%
      Anforderungsanalyse%
      \begin{enumerate}[label=\theenumi\arabic*.]%
          \item Diskussion des Problemfeldes%
          \item Konkrete Lösung%
      \end{enumerate}%
    \setcounter{enumi}{3}
    \item[\textbf{Kapitel~\arabic{enumi}:}]%
      Grundlagen und alternative Lösungen%
      \begin{enumerate}[label=\theenumi\arabic*.]%
          \item Make or Buy%
            \begin{enumerate}[label=\theenumi\arabic*.]%
                \item Nagios%
                \item ServerGuard24%
                \item PocketDBA%
            \end{enumerate}%
          \item Eigenentwicklung%
            \begin{enumerate}[label=\theenumi\arabic*.]%
                \item Vorteile einer Eigenentwicklung%
                \item Architektur%
                \item Mobile Kommunikation%
                \item Programmiersprachen%
                \item Sicherheitsaspekte%
            \end{enumerate}%
      \end{enumerate}%
    \setcounter{enumi}{4}
    \item[\textbf{Kapitel~\arabic{enumi}:}]%
      Konzeption%
      \begin{enumerate}[label=\theenumi\arabic*.]%
          \item Client-Server-Architektur%
          \item HTTPS-Server%
          \item Mobiler Client%
          \item Webclient%
      \end{enumerate}%
    \setcounter{enumi}{5}
    \item[\textbf{Kapitel~\arabic{enumi}:}]%
      Exemplarische Realisation%
      \begin{enumerate}[label=\theenumi\arabic*.]%
          \item Systemvoraussetzungen%
          \item Hard- und Software%
          \item HTTPS-Server%
          \item Mobiler Client%
          \item Webclient%
      \end{enumerate}%
    \setcounter{enumi}{6}
    \item[\textbf{Kapitel~\arabic{enumi}:}]%
      Evaluation%
    \setcounter{enumi}{7}
    \item[\textbf{Kapitel~\arabic{enumi}:}]%
      Zusammenfassung und Ausblick%
    \item[]%
      Anhänge%

  \end{enumerate}%


%---------------------------------
% Personen
%---------------------------------
\section{Personen}
\label{sec:personen}

  %---------------------------------
  % Ansprechpartner
  %---------------------------------
  \subsection{Ansprechpartner}
  \label{ssec:ansprechpartner}
    \begin{tabular}{ll}
      Name:
        &Mirko Wiechmann\\
      E-Mail:
        &\href{mailto:Mirko.Wiechmann@p3-group.com}{Mirko.Wiechmann@p3-group.com}\\
      Tel.:
        &+49 421 55 83 64 300\\
    \end{tabular}

  %---------------------------------
  % Erster Gutachter
  %---------------------------------
  \subsection{Erster Gutachter}
  \label{ssec:erster_gutachter}
    \begin{tabular}{ll}
      Name:
        &Prof. Dr. Thorsten Teschke\\
      E-Mail:
        &\href{mailto:thorsten.teschke@hs-bremen.de}{thorsten.teschke@hs-bremen.de}\\
    \end{tabular}

  %---------------------------------
  % Zweiter Gutachter
  %---------------------------------
  \subsection{Zweiter Gutachter}
  \label{ssec:zweiter_gutachter}
    \begin{tabular}{ll}
      Name:
        &{}\\
      E-Mail:
        &{}\\
    \end{tabular}

  %---------------------------------
  % Student
  %---------------------------------
  \subsection{Student}
  \label{ssec:student}
    \begin{tabular}{ll}
      Name:
        &\MetaAuthor\\
      Matrikelnr.:
        &\MetaStudentNumber\\
      E-Mail:
        &\href{mailto:roland@wolfgang-jaeger.de}{roland@wolfgang-jaeger.de}\\
      Tel.:
        &+49 163 636 43 02\\
    \end{tabular}

%---------------------------------
% Unterschriften
%---------------------------------
\section{Unterschriften}
\label{sec:unterschriften}

  \mySignatures
  {Mirko Wiechmann}
  {Prof. Dr. Thorsten Teschke}
  {Zweiter Gutachter}
  {Roland Jäger}

\end{document}
