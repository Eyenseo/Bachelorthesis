%---------------------------------
% Preamble
%---------------------------------
%---------------------------------
% Meta Variables
%---------------------------------
\newcommand{\MetaInstitute}{Hochschule Bremen}
\newcommand{\MetaUnit}{Bachelorarbeit}
\newcommand{\MetaTask}{Thesis}
\newcommand{\MetaTitle}{Konzeption und Implementierung einer Makrosprache in C++}
\newcommand{\MetaAuthorName}{Roland}
\newcommand{\MetaAuthorSurname}{J{\"a}ger}
\newcommand{\MetaStudentNumber}{360\,956}
\newcommand{\MetaAuthor}{\MetaAuthorName~\MetaAuthorSurname}

\documentclass[ngerman,a4paper,parskip=half,listof=totoc]{scrartcl}
\usepackage[T1]{fontenc} % utf8 <- produce real utf8 characters
\usepackage[utf8]{inputenc} % utf8 <- accept utf8 input characters
\usepackage{datetime}
\usepackage[ngerman]{babel}
\usepackage[automark,headsepline]{scrlayer-scrpage}
\usepackage[vscale=0.75,vmarginratio={85:100},heightrounded]{geometry} % less margin at bottom
\usepackage[svgnames]{xcolor} % svg colors
\usepackage{tablefootnote} % footnotes in tables
\usepackage{hyperref} % hyper ref "addon"
\usepackage[clean]{svg} % svg as images
\usepackage{float} % better figure placement
\usepackage{minted} % code highlighting
\usepackage{graphicx} % for pdf and other graphics
\usepackage[
  backend=biber,
  sortlocale=de_DE,
  sortcites=true,
  url=false,
  doi=true,
  eprint=false
]{biblatex}
\usepackage{csquotes} % better quotes
\usepackage{anyfontsize} % mute warnings?
\usepackage{microtype} % Subliminal refinements towards typographical perfection - eg. Hypernation
\usepackage{xspace} % set a space if not fullstop / end of sentence
\usepackage[all]{hypcap} % better linking/ref to floats
\usepackage{tikz} % drawing pretty stuff
\usepackage{ifthen} % controllflow
\usepackage{xstring} % ma­nip­u­lat­ing strings
\usepackage{calc} % calc support
\usepackage{pgfopts} % metauml / gmp
\usepackage[shellescape,latex]{gmp} % for meta uml interpretation
\usepackage[justification=centering,format=plain]{caption} % Caption setup
\usepackage{adjustbox} % Adjusts stuff if too big
\usepackage[scale=.9,ttdefault=true]{sourcecodepro}
\usepackage{enumitem} % enumerate "addon" support for named refs

\usepackage{listings} % code highlighting
\usepackage{xinttools} % for expandable and non-expandable loops
\usepackage{textcomp} % calculate support
\usepackage{lipsum}

\usepackage{paracol} % paralell collumns
\usepackage{mdframed} % Frames for text - used for breakable minted with background

\usepackage[prependcaption,colorinlistoftodos,textwidth=5.5cm,textsize=footnotesize]{todonotes}

%---------------------------------
% Add extra width to the paper for todos
%---------------------------------
\makeatletter
  \if@todonotes@disabled
  \else
    \usepackage{background}

    % draw rule for real paper width
    \SetBgScale{1}
    \SetBgAngle{0}
    \SetBgColor{lightgray}
    \SetBgContents{\rule{.4pt}\paperheight}
    \SetBgHshift{9cm}

    % add more width to the paper
    \addtolength{\paperwidth}{3cm}
  \fi
\makeatother

%---------------------------------
% TikZ libraries
%---------------------------------
\usetikzlibrary{arrows}
\usetikzlibrary{shapes}
\usetikzlibrary{spy}
\usetikzlibrary{graphs}
\usetikzlibrary{calc}
\usetikzlibrary{positioning}

%---------------------------------
% Less space below captions
%---------------------------------
\setlength{\belowcaptionskip}{-10pt}

%---------------------------------
% footnotes in footnotes helper
%---------------------------------
\makeatletter
\newcommand{\spewnotes}{%
  \tfn@tablefootnoteprintout%
  \global\let\tfn@tablefootnoteprintout\relax%
  \gdef\tfn@fnt{0}%
}
\makeatother

%---------------------------------
% Shut up I know
%---------------------------------
\pdfsuppresswarningpagegroup=1

%---------------------------------
% Biographie
%---------------------------------
\addbibresource{lit.bib}

%---------------------------------
% SVG image configuration
%---------------------------------
\setsvg{inkscape = inkscape -z -C} % better svgs

%---------------------------------
% No Page break
%---------------------------------
\newenvironment{absolutelynopagebreak}
  {\par\nobreak\vfil\penalty20\vfilneg
   \vtop\bgroup}
  {\par\xdef\tpd{\the\prevdepth}\egroup
   \prevdepth=\tpd}

%---------------------------------
% hyphen used between two capitals
%---------------------------------
\newcommand{\capitalhyphen}{%
  \raisebox{0.24ex}{%
    \resizebox{0.4em}{%
      \height%
    }{-}%
  }%
  \kern-0.07em%
}

%---------------------------------
% Minted Code
%---------------------------------
\newcommand\myCMinco[5][cpp]{
  {%
    \begin{mdframed}[innertopmargin=4pt, innerbottommargin=4pt, innerrightmargin=2pt, innerleftmargin=2pt, leftline=false, rightline=false, backgroundcolor=LightGray!20!White]
      \ifthenelse{\equal{#2}{}}{%
      }{%
        \noindent\texttt{#2}\\[-2em]%
      }%
      \inputminted[
          frame=none,
          numbersep=5pt,
          framesep=2mm,
          fontsize=\small,
          linenos,
          firstline=#4,
          lastline=#5,
          breaklines,
          breakanywhere,
        ]{#1}{#3}%
    \end{mdframed}
  }
}
\newcommand\myMinco[5][cpp]{
  {%
    \begin{mdframed}[innertopmargin=4pt, innerbottommargin=4pt, innerrightmargin=2pt, innerleftmargin=2pt, leftline=false, rightline=false]
      \ifthenelse{\equal{#2}{}}{%
      }{%
        \noindent\texttt{#2}\\[-2em]%
      }%
      \inputminted[
          numbersep=5pt,
          framesep=2mm,
          fontsize=\small,
          linenos,
          firstline=#4,
          lastline=#5,
          breaklines,
          breakanywhere,
        ]{#1}{#3}%
    \end{mdframed}
  }
}
\newcommand\myCMincoLine[5][cpp]{
  {%
    \begin{mdframed}[innertopmargin=4pt, innerbottommargin=4pt, innerrightmargin=2pt, innerleftmargin=2pt, leftline=false, rightline=false, backgroundcolor=LightGray!20!White]
      \ifthenelse{\equal{#2}{}}{%
      }{%
        \noindent\texttt{#2}\\[-2em]%
      }%
      \inputminted[
          numbersep=5pt,
          framesep=2mm,
          fontsize=\small,
          linenos,
          firstnumber=#4,
          firstline=#4,
          lastline=#5,
        ]{#1}{#3}%
    \end{mdframed}
  }
}
\newcommand\myMincoLine[5][cpp]{
  {%
    \begin{mdframed}[innertopmargin=4pt, innerbottommargin=4pt, innerrightmargin=2pt, innerleftmargin=2pt, leftline=false, rightline=false]
      \ifthenelse{\equal{#2}{}}{%
      }{%
        \noindent\texttt{#2}\\[-2em]%
      }%
      \inputminted[
          numbersep=5pt,
          framesep=2mm,
          fontsize=\small,
          linenos,
          firstnumber=#4,
          firstline=#4,
          lastline=#5,
        ]{#1}{#3}%
    \end{mdframed}
  }
}

%---------------------------------
% Minted inline
%---------------------------------
\newcommand{\myMinin}[1][]{\mintinline[#1]{c++}}

%---------------------------------
% Horizontal line for title page
%---------------------------------
\newcommand{\HRule}{\rule{\linewidth}{0.2mm}}

%---------------------------------
% href Link as foot note
%---------------------------------
\newcommand\myFnurl[2]{%
  \ifthenelse{\equal{#1}{}}{%
    \footnote{\url{#2}}%
  }{%
    \href{#2}{#1}\footnote{ \url{#2}}%
  }%
}

%---------------------------------
% Table in table for new line
%---------------------------------
\newcommand{\specialcell}[2][c]{% acceps t, b and c for vertival alignment
  \begin{tabular}[#1]{@{}l@{}}#2\end{tabular}}

%---------------------------------
% Meta data and Link Colour
%---------------------------------
\newcommand\myShade{70}

\definecolor{mylinkcolor}{RGB}{113, 31, 155}
\definecolor{mycitecolor}{RGB}{255, 189, 76}
\definecolor{myurlcolor}{RGB}{62, 106, 171}

\hypersetup{
  pdfauthor   = {\MetaAuthor},
  pdftitle    = {\MetaTitle},
  pdfsubject  = {\MetaUnit, \MetaTask},
  % pdfsubject  = {\MetaUnit},
  % pdfkeywords = {\MetaTitle, \MetaUnit, \MetaInstitute},
  pdfkeywords = {\MetaTitle, \MetaUnit, \MetaTask, \MetaInstitute},
  colorlinks  = true,
  linkcolor   = mylinkcolor!\myShade!black,
  citecolor   = mycitecolor!\myShade!black,
  urlcolor    = myurlcolor!\myShade!black,
}

%---------------------------------
% Minted color scheme
%---------------------------------
\usemintedstyle{borland}

%---------------------------------
% Fancy header
%---------------------------------
\clearpairofpagestyles
\lohead{\headmark}
\cofoot[\pagemark]{\pagemark}
\pagestyle{scrheadings}

%---------------------------------
% Start page count
%---------------------------------
\setcounter{page}{0}

%---------------------------------
% todo notes
%---------------------------------
\newcommand{\myTodo}[2][NOCAP]{
  \ifthenelse{\equal{#1}{NOCAP}}{%
    \todo[color=SandyBrown]{#2}%
  }{%
    \todo[color=SandyBrown, caption={#1}]{#2}%
  }%
}
\newcommand{\myFixme}[2][NOCAP]{
  \ifthenelse{\equal{#1}{NOCAP}}{%
    \todo[color=FireBrick!80]{#2}%
  }{%
    \todo[color=FireBrick!80, caption={#1}]{#2}%
  }%
}
\newcommand{\myQuestion}[2][NOCAP]{
  \ifthenelse{\equal{#1}{NOCAP}}{%
    \todo[color=LightBlue]{#2}%
  }{%
    \todo[color=LightBlue, caption={#1}]{#2}%
  }%
}

%---------------------------------
% Reference sections by name and number
%---------------------------------
\addto\extrasngerman{
  \def\sectionautorefname{Ka\-pi\-tel}
  \def\subsectionautorefname{Un\-ter\-ka\-pi\-tel}
}
\newcommand{\myNamedRef}[1]{%
  \hyperref[#1]{%
    \autoref*{#1} \nameref*{#1}%
  }%
}

%---------------------------------
% Listings for Macro syntax
%---------------------------------
\definecolor{Code}{rgb}{0,0,0}
\definecolor{Keywords}{rgb}{0,0,0.6}
\definecolor{Strings}{rgb}{0,0,1}
\definecolor{Comments}{rgb}{.5,.5,.5}
\definecolor{Numbers}{rgb}{0,0,1}


\makeatletter
\newif\iffirstchar\firstchartrue
\newif\ifstartedbyadigit

\lst@AddToHook{Output}%
{%
  \ifstartedbyadigit
    \def\lst@thestyle{\color{Numbers}}%
  \fi
  \global\firstchartrue
  \global\startedbyadigitfalse
}
\newcommand\processletter
{%
  \ifnum\lst@mode=\lst@Pmode
    \iffirstchar%
        \global\startedbyadigitfalse
      \fi
      \global\firstcharfalse
    \fi
}
\newcommand\processdigit
{%
  \ifnum\lst@mode=\lst@Pmode
      \iffirstchar
        \global\startedbyadigittrue
      \fi
      \global\firstcharfalse
  \fi
}
\newcommand\addtoletterdef[2]
{%
  \expandafter\lst@DefSaveDef
  \expandafter{%
  \expandafter`%
  \expandafter#2%
  \expandafter}%
  \csname jubobs@#2\expandafter\endcsname
  \expandafter{\csname jubobs@#2\endcsname #1}%
}
\makeatother

\lstdefinelanguage{MyMacro}{
  keywords={var, def, return, if, while, do, else, break, for},
  ndkeywords={true, false},
  ndkeywordstyle=\color{Numbers},
  sensitive=false,
  comment=[l]{//},
  morecomment=[s]{/*}{*/},
  commentstyle=\color{purple}\ttfamily,
  morestring=[b]"
}
\lstdefinestyle{MyMacroStyle}
{
  language=MyMacro,
  keepspaces,
  commentstyle=\color{Comments}\slshape,
  stringstyle=\color{Strings},
  keywordstyle={\color{Keywords}\bfseries},
  alsoletter=0123456789,
  SelectCharTable=%
      \xintApplyInline{\addtoletterdef\processdigit}{0123456789}%
      \xintApplyInline{\addtoletterdef\processletter}{abcdefghijklmnopqrstuvwxyzABCDEFGHIJKLMNOPQRSTUVWXYZ}%
}
\lstdefinelanguage{MyRegex}{
  alsoletter={\\,*,\&,^},
  basicstyle=\color{OliveDrab},
  keywords={\\,\\.},
  morekeywords=[2]{\\d,\\s},
  keywordstyle=*\color{OliveDrab},
  keywordstyle=[2]\color{DarkOrange},
  sensitive=false,
  postbreak=,
  literate={[}{{\textcolor{DarkOrange}{[}}}{1}
           {]}{{\textcolor{DarkOrange}{]}}}{1}
           {*}{{\textcolor{DarkOrange}{*}}}{1}
           {.}{{\textcolor{DarkOrange}{.}}}{1}
           {+}{{\textcolor{DarkOrange}{+}}}{1}
           {?}{{\textcolor{DarkOrange}{?}}}{1}
           {-}{{\textcolor{DarkOrange}{-}}}{1}
           {(}{{\textcolor{DarkOrange}{(}}}{1}
           {)}{{\textcolor{DarkOrange}{)}}}{1}
           {\^}{{\textcolor{DarkOrange}{\textasciicircum{}}}}{1}
           {\\.}{{\textcolor{OliveDrab}{\textbackslash.}}}{1}
}
\lstset{
  postbreak=\raisebox{0ex}[0ex][0ex]{\ensuremath{\color{red}\hookrightarrow\space}},
  backgroundcolor=\color{LightGray!20!White},
  extendedchars=true,
  basicstyle=\small\ttfamily,
  showstringspaces=false,
  showspaces=false,
  numbers=left,
  numberstyle=\tiny,
  numbersep=4pt,
  tabsize=2,
  breaklines=true,
  showtabs=false,
  captionpos=b,
  frame=lines,
  columns=fixed,
  basewidth={.56em,.45em},
  fontadjust=true,
}

%---------------------------------
% Tikz Railroad styles
%---------------------------------
\definecolor{nonterminal_color}{RGB}{255, 230, 179}
\definecolor{terminal_color}{RGB}{230, 230, 230}
\tikzset{
  nonterminal/.style = {
    rectangle,
    rounded corners=3pt,
    minimum size=6mm,
    draw=nonterminal_color!80!black!95,
    fill=nonterminal_color,
    text height=1.5ex,
    text depth=.25ex,
    font=\ttfamily,
  },
  terminal/.style = {
    rounded rectangle,
    minimum size=6mm,
    draw=terminal_color!80!black!95,
    fill=terminal_color,
    text height=1.5ex,
    text depth=.25ex,
    font=\ttfamily,
  },
  StartEnd/.style = {
    circle,
    minimum size=3mm,
    draw=black!80,
    fill=black!60,
  },
  skip loop/.style = {
    to path={-- ++(0,#1) -| (\tikztotarget)}
  },
  hv path/.style = {
    to path={-| (\tikztotarget)}
  },
  vh path/.style = {
    to path={|- (\tikztotarget)}
  },
  railroad/.style = {
    >=stealth',
    black!50,
    text=black,
    thick,
    node distance = 4mm
  },
  graphs/railroad/.style = {
    edges=rounded corners,
    simple,
  }
}

%---------------------------------
% An option to toggle inputs - used for tikz graphics
%---------------------------------
\newcommand{\myInput}[1]{%
  \adjustbox{max width=.9\linewidth}{%
    \scalebox{.9}{%
      \input{#1}%
    }%
  }%
}%
\newcommand{\myInputUnlimited}[1]{%
  \scalebox{.9}{%
    \input{#1}%
  }%
}%

%---------------------------------
% An Environment for code with captions
%---------------------------------
\DeclareCaptionType{myCodeEnvType}[Code][Quell\-code\-ver\-zei\-chnis]
\newenvironment{myCodeEnv}{
    \vspace{1em}%
    \captionsetup{type=myCodeEnvType}%
  }{%
}



%---------------------------------
% Document start
%---------------------------------
\begin{document}
%---------------------------------
% Titlepage
%---------------------------------
\begin{titlepage}
  \shortdate % Use Short Date
  \center % Center everything on the page

  %---------------------------------
  % HEADING SECTIONS
  %---------------------------------
  ~\\[2cm]
  % ~\\[2.5cm]
  \textsc{\LARGE \MetaInstitute}\\[1.5cm] % Name of your university/college
  \textsc{\Large \MetaUnit}\\[0.5cm] % Major heading such as course name
  \textsc{\large \MetaTask}\\[2cm] % Minor heading such as course title

  %---------------------------------
  % TITLE SECTION
  %---------------------------------
  \HRule \\[0.5cm]
  {
    \LARGE \bfseries \MetaTitle \\[0.50cm] % Title of your document
    \par
  }
  \HRule \\[1.5cm]

  %---------------------------------
  % AUTHOR SECTION
  %---------------------------------
  \begin{minipage}{0.4\textwidth}
    \begin{flushleft}
      \large
        \emph{Author:}\\
        \MetaAuthorName~\textsc{\MetaAuthorSurname}\\
      \small \MetaStudentNumber
    \end{flushleft}
  \end{minipage}
  ~
  \begin{minipage}{0.4\textwidth}
    \begin{flushright}

    \end{flushright}
  \end{minipage}\\[4cm]

  %---------------------------------
  % DATE SECTION
  %---------------------------------
  \vspace*{\fill}
  {
    \large \today
  }
\end{titlepage}


%---------------------------------
% table of contents
%---------------------------------
\tableofcontents
\newpage

%---------------------------------
% Einleitung
%---------------------------------
\section{Einleitung}
\label{sec:einleitung}

%---------------------------------
% Problemstellung
%---------------------------------
\section{Problemstellung}
\label{sec:problemstellung}

%---------------------------------
% Lösungsansatz
%---------------------------------
\section{Lösungsansatz}
\label{sec:loesungsansatz}

%---------------------------------
% Konkrete Aufgaben
%---------------------------------
\section{Konkrete Aufgaben}
\label{sec:konkrete_aufgaben}

%---------------------------------
% Arbeitsumfeld
%---------------------------------
\section{Arbeitsumfeld}
\label{sec:arbeitsumfeld}

  %---------------------------------
  % Literatur
  %---------------------------------
  \subsection{Literatur}
  \label{ssec:literatur}
  \nocite{Johnson:1997:FRA:262793.262799}
  \bibliography{lit}

  %---------------------------------
  % Software
  %---------------------------------
  \subsection{Software}
  \label{ssec:software}

  %---------------------------------
  % Hardware
  %---------------------------------
  \subsection{Hardware}
  \label{ssec:hardware}


%---------------------------------
% Planung
%---------------------------------
\section{Planung}
\label{sec:planung}

  %---------------------------------
  % Wann
  %---------------------------------
  \subsection{Wann}
  \label{ssec:wann}
    März bis Juni 2015

  %---------------------------------
  % Wo
  %---------------------------------
  \subsection{Wo}
  \label{ssec:wo}
    \begin{tabular}{l}
      P3 engineering GmbH\\
      Flughafenallee 26/28\\
      28199 Bremen\\
      \href{www.p3-group.com}{www.p3-group.com}\\
    \end{tabular}

  %---------------------------------
  % Arbeitspakete
  %---------------------------------
  \subsection{Arbeitspakete}
  \label{ssec:arbeitspakete}
  \begin{itemize}%
    \item \myRoadMapTime{Recherche}{ca.1 Woche}%
    \item \myRoadMapTime{Konzeption}{ca.1 Woche}%
      \begin{itemize}%
        \item \myRoadMapTime{Serverarchitektur / Kontrollmechanismen}{}%
        \item \myRoadMapTime{Mobiler Client}{}%
        \item \myRoadMapTime{Webclient}{}%
      \end{itemize}%
    \item \myRoadMapTime{Implementierung inkl. Dokumentation}{ca.1 Woche}%
      \begin{itemize}%
        \item \myRoadMapTime{Serverarchitektur / Kontrollmechanismen}{ca.1 Woche}%
        \item \myRoadMapTime{Mobiler Client}{ca.1 Woche}%
        \item \myRoadMapTime{Webclient}{ca.1 Woche}%
      \end{itemize}%
    \item \myRoadMapTime{Testen}{ca.1 Woche}%
    \item \myRoadMapTime{Verfassen der BT}{ca.1 Woche}%
      \begin{itemize}%
        \item \myRoadMapTime{Einleitung + Anforderungsanalyse}{ca. 1 Woche, ab 10. Oktober}%
          \begin{itemize}%
            \item \myRoadMapTime{Kapitel 1: Allgemeines}{}%
            \item \myRoadMapTime{Kapitel 2: Einleitung}{}%
            \item \myRoadMapTime{Kapitel 3: Anforderungsanalyse}{}%
          \end{itemize}%
        \item \myRoadMapTime{Hauptteil}{ca. 2,5 Wochen, ab 17. Oktober}%
          \begin{itemize}%
            \item \myRoadMapTime{Kapitel 4: Grundlagen und alternative Lösungen}{}%
            \item \myRoadMapTime{Kapitel 5: Konzeption}{}%
            \item \myRoadMapTime{Kapitel 6: Exemplarische Realisation}{}%
          \end{itemize}%
        \item \myRoadMapTime{Schlussteil}{< 1 Woche, ab 7. November}%
          \begin{itemize}%
            \item \myRoadMapTime{Kapitel 7: Zusammenfassung und Ausblick}{}%
            \item \myRoadMapTime{Kapitel 8: Anhang}{}%
          \end{itemize}%
      \end{itemize}%
  \end{itemize}%

  %---------------------------------
  % Meilensteine
  %---------------------------------
  \subsection{Meilensteine}
  \label{ssec:meilensteine}

    \begin{table}[H]
      \begin{tabularx}{\textwidth}{lX}%
        \textbf{22. August:}        & Abschluss der Recherche / Beginn der Konzeption\\%
        \textbf{29. August:}        & Abschluss der Konzeption / Beginn der Implementierung\\%
        \textbf{10. Oktober:}        & Abschluss der Implementierung, Dokumentation und Testphase / Beginn der schriftlichen Arbeit\\%
        \textbf{17. Oktober:}        & Beginn der schriftlichen Arbeit am Hauptteil\\%
        \textbf{10. November:}        & Abschluss der schriftlichen Arbeit / Beginn der Korrektur, Binden der DA etc.\\%
        \textbf{15. November:}        & Abgabe der Arbeit\\%
      \end{tabularx}
    \end{table}

%---------------------------------
% Gliederung der Arbeit
%---------------------------------
\section{Gliederung der Arbeit}
\label{sec:gliederung_der_arbeit}
  \begin{enumerate}[
    labelindent=*,
    % style=multiline,
    leftmargin=\widthof{\textbf{Kapitel~0:~}},
    label=\arabic*.
  ]%
    \item[]%
      Allgemeines%
      \begin{enumerate}[label=\theenumi\arabic*.]%
          \item[]%
            Eidesstattliche Erklärung%
          \item[]%
            Danksagung%
      \end{enumerate}%
    \setcounter{enumi}{1}
    \item[\textbf{Kapitel~\arabic{enumi}:}]%
      Einleitung%
      \begin{enumerate}[label=\theenumi\arabic*.]%
          \item Problemfeld%
          \item Ziele der Arbeit%
          \item Hintergründe und Entstehung des Themas%
          \item Struktur der Arbeit, wesentliche Inhalte der Kapitel%
      \end{enumerate}%
    \setcounter{enumi}{2}
    \item[\textbf{Kapitel~\arabic{enumi}:}]%
      Anforderungsanalyse%
      \begin{enumerate}[label=\theenumi\arabic*.]%
          \item Diskussion des Problemfeldes%
          \item Konkrete Lösung%
      \end{enumerate}%
    \setcounter{enumi}{3}
    \item[\textbf{Kapitel~\arabic{enumi}:}]%
      Grundlagen und alternative Lösungen%
      \begin{enumerate}[label=\theenumi\arabic*.]%
          \item Make or Buy%
            \begin{enumerate}[label=\theenumi\arabic*.]%
                \item Nagios%
                \item ServerGuard24%
                \item PocketDBA%
            \end{enumerate}%
          \item Eigenentwicklung%
            \begin{enumerate}[label=\theenumi\arabic*.]%
                \item Vorteile einer Eigenentwicklung%
                \item Architektur%
                \item Mobile Kommunikation%
                \item Programmiersprachen%
                \item Sicherheitsaspekte%
            \end{enumerate}%
      \end{enumerate}%
    \setcounter{enumi}{4}
    \item[\textbf{Kapitel~\arabic{enumi}:}]%
      Konzeption%
      \begin{enumerate}[label=\theenumi\arabic*.]%
          \item Client-Server-Architektur%
          \item HTTPS-Server%
          \item Mobiler Client%
          \item Webclient%
      \end{enumerate}%
    \setcounter{enumi}{5}
    \item[\textbf{Kapitel~\arabic{enumi}:}]%
      Exemplarische Realisation%
      \begin{enumerate}[label=\theenumi\arabic*.]%
          \item Systemvoraussetzungen%
          \item Hard- und Software%
          \item HTTPS-Server%
          \item Mobiler Client%
          \item Webclient%
      \end{enumerate}%
    \setcounter{enumi}{6}
    \item[\textbf{Kapitel~\arabic{enumi}:}]%
      Evaluation%
    \setcounter{enumi}{7}
    \item[\textbf{Kapitel~\arabic{enumi}:}]%
      Zusammenfassung und Ausblick%
    \item[]%
      Anhänge%

  \end{enumerate}%


%---------------------------------
% Personen
%---------------------------------
\section{Personen}
\label{sec:personen}

  %---------------------------------
  % Ansprechpartner
  %---------------------------------
  \subsection{Ansprechpartner}
  \label{ssec:ansprechpartner}
    \begin{tabular}{ll}
      Name:
        &Mirko Wiechmann\\
      E-Mail:
        &\href{mailto:Mirko.Wiechmann@p3-group.com}{Mirko.Wiechmann@p3-group.com}\\
      Tel.:
        &+49 421 55 83 64 300\\
    \end{tabular}

  %---------------------------------
  % Erster Gutachter
  %---------------------------------
  \subsection{Erster Gutachter}
  \label{ssec:erster_gutachter}
    \begin{tabular}{ll}
      Name:
        &Prof. Dr. Thorsten Teschke\\
      E-Mail:
        &\href{mailto:thorsten.teschke@hs-bremen.de}{thorsten.teschke@hs-bremen.de}\\
    \end{tabular}

  %---------------------------------
  % Zweiter Gutachter
  %---------------------------------
  \subsection{Zweiter Gutachter}
  \label{ssec:zweiter_gutachter}
    \begin{tabular}{ll}
      Name:
        &{}\\
      E-Mail:
        &{}\\
    \end{tabular}

  %---------------------------------
  % Student
  %---------------------------------
  \subsection{Student}
  \label{ssec:student}
    \begin{tabular}{ll}
      Name:
        &\MetaAuthor\\
      Matrikelnr.:
        &\MetaStudentNumber\\
      E-Mail:
        &\href{mailto:roland@wolfgang-jaeger.de}{roland@wolfgang-jaeger.de}\\
      Tel.:
        &+49 163 636 43 02\\
    \end{tabular}

%---------------------------------
% Unterschriften
%---------------------------------
\section{Unterschriften}
\label{sec:unterschriften}

  \mySignatures
  {Mirko Wiechmann}
  {Prof. Dr. Thorsten Teschke}
  {Zweiter Gutachter}
  {Roland Jäger}

\end{document}
