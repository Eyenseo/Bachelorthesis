%---------------------------------
% Meta Variables
%---------------------------------
\newcommand{\MetaInstitute}{Hochschule Bremen}
\newcommand{\MetaUnit}{Bachelorarbeit}
\newcommand{\MetaTask}{Thesis}
\newcommand{\MetaTitle}{Konzeption und Implementierung einer Makrosprache in C++}
\newcommand{\MetaAuthorName}{Roland}
\newcommand{\MetaAuthorSurname}{J{\"a}ger}
\newcommand{\MetaStudentNumber}{360\,956}
\newcommand{\MetaAuthor}{\MetaAuthorName~\MetaAuthorSurname}

\documentclass[ngerman,a4paper,parskip=half,toc=listof]{scrartcl}
\usepackage[T1]{fontenc} % utf8 <- produce real utf8 characters
\usepackage[utf8]{inputenc} % utf8 <- accept utf8 input characters
\usepackage{datetime}
\usepackage[ngerman]{babel}
\usepackage[automark,headsepline]{scrlayer-scrpage}
\usepackage[vscale=0.75,vmarginratio={85:100},heightrounded]{geometry} % less margin at bottom
\usepackage[svgnames]{xcolor} % svg colors
\usepackage{tablefootnote} % footnotes in tables
\usepackage{hyperref} % hyper ref "addon"
% \usepackage[clean]{svg} % svg as images % BUG no trancparency
\usepackage{float} % better figure placement
\usepackage{minted} % code highlighting
\usepackage{graphicx} % for pdf and other graphics
\usepackage[
  backend=biber,
  sortlocale=de_DE,
  sortcites=true,
  url=false,
  doi=true,
  eprint=false
]{biblatex}
\usepackage{csquotes} % better quotes
\usepackage{anyfontsize} % mute warnings?
\usepackage[activate={true,nocompatibility},final,tracking=true,kerning=true,spacing=true,factor=1100,stretch=10,shrink=10]{microtype}  % Subliminal refinements towards typographical perfection - eg. Hypernation
% activate={true,nocompatibility} - activate protrusion and expansion
% final - enable microtype; use "draft" to disable
% tracking=true, kerning=true, spacing=true - activate these techniques
% factor=1100 - add 10% to the protrusion amount (default is 1000)
% stretch=10, shrink=10 - reduce stretchability/shrinkability (default is 20/20)
\usepackage{xspace} % set a space if not fullstop / end of sentence
\usepackage[all]{hypcap} % better linking/ref to floats
\usepackage{tikz} % drawing pretty stuff
\usepackage{ifthen} % controllflow
\usepackage{xstring} % ma­nip­u­lat­ing strings
\usepackage{calc} % calc support
\usepackage{pgfopts} % metauml / gmp
\usepackage[shellescape,latex]{gmp} % for meta uml interpretation
\usepackage[justification=centering,format=plain]{caption} % Caption setup
\usepackage{adjustbox} % Adjusts stuff if too big
\usepackage[scale=.9,ttdefault=true]{sourcecodepro}
\usepackage{enumitem} % enumerate "addon" support for named refs

\usepackage{listings} % code highlighting
\usepackage{xinttools} % for expandable and non-expandable loops
\usepackage{textcomp} % calculate support
\usepackage{lipsum}
\usepackage{array} % for better tables

\usepackage{paracol} % paralell collumns
\usepackage{mdframed} % Frames for text - used for breakable minted with background

\usepackage[prependcaption,colorinlistoftodos,textwidth=5.5cm,textsize=footnotesize]{todonotes}

\usepackage[many]{tcolorbox} % used to wrapp listings invisible - can produce nice colored boxes thou

%---------------------------------
% Add extra width to the paper for todos
%---------------------------------
\makeatletter
  \if@todonotes@disabled
  \else
    \usepackage{background}

    % draw rule for real paper width
    \SetBgScale{1}
    \SetBgAngle{0}
    \SetBgColor{lightgray}
    \SetBgContents{\rule{.4pt}\paperheight}
    \SetBgHshift{9cm}

    % add more width to the paper
    \addtolength{\paperwidth}{3cm}
  \fi
\makeatother

%---------------------------------
% TikZ libraries
%---------------------------------
\usetikzlibrary{arrows}
\usetikzlibrary{shapes}
\usetikzlibrary{spy}
\usetikzlibrary{graphs}
\usetikzlibrary{calc}
\usetikzlibrary{positioning}

%---------------------------------
% Less space below captions
%---------------------------------
\setlength{\belowcaptionskip}{-10pt}

%---------------------------------
% footnotes in footnotes helper
%---------------------------------
\makeatletter
\newcommand{\spewnotes}{%
  \tfn@tablefootnoteprintout%
  \global\let\tfn@tablefootnoteprintout\relax%
  \gdef\tfn@fnt{0}%
}
\makeatother

%---------------------------------
% Shut up I know
%---------------------------------
\pdfsuppresswarningpagegroup=1

%---------------------------------
% Biographie
%---------------------------------
\addbibresource{lit.bib}

%---------------------------------
% SVG image configuration
%---------------------------------
% \setsvg{inkscape = inkscape -z -C} % better svgs

%---------------------------------
% No Page break
%---------------------------------
\newenvironment{absolutelynopagebreak}
  {\par\nobreak\vfil\penalty20\vfilneg
   \vtop\bgroup}
  {\par\xdef\tpd{\the\prevdepth}\egroup
   \prevdepth=\tpd}

%---------------------------------
% hyphen used between two capitals
%---------------------------------
\newcommand{\capitalhyphen}{%
  \raisebox{0.24ex}{%
    \resizebox{0.4em}{%
      \height%
    }{-}%
  }%
  \kern-0.07em%
}

%---------------------------------
% Minted Code
%---------------------------------
\newcommand\myCMinco[5][cpp]{
  {%
    \begin{mdframed}[innertopmargin=4pt, innerbottommargin=4pt, innerrightmargin=2pt, innerleftmargin=2pt, leftline=false, rightline=false, backgroundcolor=LightGray!20!White]
      \ifthenelse{\equal{#2}{}}{%
      }{%
        \noindent\texttt{#2}\\[-2em]%
      }%
      \inputminted[
          frame=none,
          numbersep=5pt,
          framesep=2mm,
          fontsize=\small,
          linenos,
          firstline=#4,
          lastline=#5,
          breaklines,
          breakanywhere,
        ]{#1}{#3}%
    \end{mdframed}
  }
}
\newcommand\myMinco[5][cpp]{
  {%
    \begin{mdframed}[innertopmargin=4pt, innerbottommargin=4pt, innerrightmargin=2pt, innerleftmargin=2pt, leftline=false, rightline=false]
      \ifthenelse{\equal{#2}{}}{%
      }{%
        \noindent\texttt{#2}\\[-2em]%
      }%
      \inputminted[
          numbersep=5pt,
          framesep=2mm,
          fontsize=\small,
          linenos,
          firstline=#4,
          lastline=#5,
          breaklines,
          breakanywhere,
        ]{#1}{#3}%
    \end{mdframed}
  }
}
\newcommand\myCMincoLine[5][cpp]{
  {%
    \begin{mdframed}[innertopmargin=4pt, innerbottommargin=4pt, innerrightmargin=2pt, innerleftmargin=2pt, leftline=false, rightline=false, backgroundcolor=LightGray!20!White]
      \ifthenelse{\equal{#2}{}}{%
      }{%
        \noindent\texttt{#2}\\[-2em]%
      }%
      \inputminted[
          numbersep=5pt,
          framesep=2mm,
          fontsize=\small,
          linenos,
          firstnumber=#4,
          firstline=#4,
          lastline=#5,
        ]{#1}{#3}%
    \end{mdframed}
  }
}
\newcommand\myMincoLine[5][cpp]{
  {%
    \begin{mdframed}[innertopmargin=4pt, innerbottommargin=4pt, innerrightmargin=2pt, innerleftmargin=2pt, leftline=false, rightline=false]
      \ifthenelse{\equal{#2}{}}{%
      }{%
        \noindent\texttt{#2}\\[-2em]%
      }%
      \inputminted[
          numbersep=5pt,
          framesep=2mm,
          fontsize=\small,
          linenos,
          firstnumber=#4,
          firstline=#4,
          lastline=#5,
        ]{#1}{#3}%
    \end{mdframed}
  }
}

%---------------------------------
% Minted inline
%---------------------------------
\newcommand{\myMinin}[1][breaklines,breakanywhere,breakanywheresymbolpre=]{\mintinline[#1]{c++}}
\newcommand{\myMin}[1][breaklines,breakanywhere,breakanywheresymbolpre=]{\mintinline[#1]{text}}

%---------------------------------
% Horizontal line for title page
%---------------------------------
\newcommand{\HRule}{\rule{\linewidth}{0.2mm}}

%---------------------------------
% href Link as foot note
%---------------------------------
\newcommand\myFnurl[2]{%
  \ifthenelse{\equal{#1}{}}{%
    \footnote{\url{#2}}%
  }{%
    \href{#2}{#1}\footnote{ \url{#2}}%
  }%
}

%---------------------------------
% Table in table for new line
%---------------------------------
\newcommand{\specialcell}[2][c]{% acceps t, b and c for vertival alignment
  \begin{tabular}[#1]{@{}l@{}}#2\end{tabular}}

%---------------------------------
% Meta data and Link Colour
%---------------------------------
\newcommand\myShade{70}

\definecolor{mylinkcolor}{RGB}{113, 31, 155}
\definecolor{mycitecolor}{RGB}{255, 189, 76}
\definecolor{myurlcolor}{RGB}{62, 106, 171}

\hypersetup{
  pdfauthor   = {\MetaAuthor},
  pdftitle    = {\MetaTitle},
  pdfsubject  = {\MetaUnit, \MetaTask},
  % pdfsubject  = {\MetaUnit},
  % pdfkeywords = {\MetaTitle, \MetaUnit, \MetaInstitute},
  pdfkeywords = {\MetaTitle, \MetaUnit, \MetaTask, \MetaInstitute},
  colorlinks  = true,
  linkcolor   = mylinkcolor!\myShade!black,
  citecolor   = mycitecolor!\myShade!black,
  urlcolor    = myurlcolor!\myShade!black,
}

%---------------------------------
% Minted color scheme
%---------------------------------
\usemintedstyle{borland}

%---------------------------------
% Fancy header
%---------------------------------
\clearpairofpagestyles
\lohead{\headmark}
\cofoot[\pagemark]{\pagemark}
\pagestyle{scrheadings}

%---------------------------------
% Start page count
%---------------------------------
\setcounter{page}{0}

%---------------------------------
% todo notes
%---------------------------------
\newcommand{\myTodo}[2][NOCAP]{
  \ifthenelse{\equal{#1}{NOCAP}}{%
    \todo[color=SandyBrown]{#2}%
  }{%
    \todo[color=SandyBrown, caption={#1}]{#2}%
  }%
}
\newcommand{\myFixme}[2][NOCAP]{
  \ifthenelse{\equal{#1}{NOCAP}}{%
    \todo[color=FireBrick!80]{#2}%
  }{%
    \todo[color=FireBrick!80, caption={#1}]{#2}%
  }%
}
\newcommand{\myQuestion}[2][NOCAP]{
  \ifthenelse{\equal{#1}{NOCAP}}{%
    \todo[color=LightBlue]{#2}%
  }{%
    \todo[color=LightBlue, caption={#1}]{#2}%
  }%
}

%---------------------------------
% Reference sections by name and number
%---------------------------------
\addto\extrasngerman{
  \def\sectionautorefname{Ka\-pi\-tel}
  \def\subsectionautorefname{Un\-ter\-ka\-pi\-tel}
  \def\subsubsectionautorefname{Un\-ter\-un\-ter\-ka\-pi\-tel}
}
\newcommand{\myNamedRef}[1]{%
  \hyperref[#1]{%
    \autoref*{#1} \nameref*{#1}%
  }%
}

%---------------------------------
% Listings for Macro syntax
%---------------------------------
\definecolor{Code}{rgb}{0,0,0}
\definecolor{Keywords}{rgb}{0,0,0.6}
\definecolor{Strings}{rgb}{0,0,1}
\definecolor{Comments}{rgb}{.5,.5,.5}
\definecolor{Numbers}{rgb}{0,0,1}


\makeatletter
\newif\iffirstchar\firstchartrue
\newif\ifstartedbyadigit

\lst@AddToHook{Output}%
{%
  \ifstartedbyadigit
    \def\lst@thestyle{\color{Numbers}}%
  \fi
  \global\firstchartrue
  \global\startedbyadigitfalse
}
\newcommand\processletter
{%
  \ifnum\lst@mode=\lst@Pmode
    \iffirstchar%
        \global\startedbyadigitfalse
      \fi
      \global\firstcharfalse
    \fi
}
\newcommand\processdigit
{%
  \ifnum\lst@mode=\lst@Pmode
      \iffirstchar
        \global\startedbyadigittrue
      \fi
      \global\firstcharfalse
  \fi
}
\newcommand\addtoletterdef[2]
{%
  \expandafter\lst@DefSaveDef
  \expandafter{%
  \expandafter`%
  \expandafter#2%
  \expandafter}%
  \csname jubobs@#2\expandafter\endcsname
  \expandafter{\csname jubobs@#2\endcsname #1}%
}
\makeatother

\lstdefinelanguage{MyMacro}{
  keywords={var, def, return, if, while, do, else, break, for, print, typeof},
  ndkeywords={true, false},
  ndkeywordstyle=\color{Numbers},
  sensitive=false,
  comment=[l]{//},
  morecomment=[s]{/*}{*/},
  commentstyle=\color{purple}\ttfamily,
  morestring=[b]"
}
\lstdefinestyle{MyMacroStyle}
{
  language=MyMacro,
  keepspaces,
  commentstyle=\color{Comments}\slshape,
  stringstyle=\color{Strings},
  keywordstyle={\color{Keywords}\bfseries},
  alsoletter=0123456789.,
  SelectCharTable=%
      \xintApplyInline{\addtoletterdef\processdigit}{0123456789}%
      \xintApplyInline{\addtoletterdef\processletter}{abcdefghijklmnopqrstuvwxyzABCDEFGHIJKLMNOPQRSTUVWXYZ}%
}
\lstdefinelanguage{MyRegex}{
  alsoletter={\\,*,\&,^},
  basicstyle=\color{DarkGreen},
  keywords={\\,\\.,\\(,\\)},
  morekeywords=[2]{\\d,\\s},
  keywordstyle=*\color{DarkGreen},
  keywordstyle=[2]\color{DarkOrange},
  sensitive=false,
  postbreak=,
  literate={[}{{\textcolor{DarkOrange}{[}}}{1}
           {]}{{\textcolor{DarkOrange}{]}}}{1}
           {*}{{\textcolor{DarkOrange}{*}}}{1}
           {.}{{\textcolor{DarkOrange}{.}}}{1}
           {+}{{\textcolor{DarkOrange}{+}}}{1}
           {?}{{\textcolor{DarkOrange}{?}}}{1}
           {-}{{\textcolor{DarkOrange}{-}}}{1}
           {(}{{\textcolor{DarkOrange}{(}}}{1}
           {)}{{\textcolor{DarkOrange}{)}}}{1}
           {\{}{{\textcolor{DarkOrange}{\{}}}{1}
           {\}}{{\textcolor{DarkOrange}{\}}}}{1}
           {|}{{\textcolor{DarkOrange}{|}}}{1}
           {\^}{{\textcolor{DarkOrange}{\textasciicircum{}}}}{1}
           {\\\{}{\textcolor{DarkOrange}{\textbackslash}{\textcolor{DarkGreen}{\{}}}{1}
           {\\\}}{\textcolor{DarkOrange}{\textbackslash}{\textcolor{DarkGreen}{\}}}}{1}
           {\\(}{\textcolor{DarkOrange}{\textbackslash}{\textcolor{DarkGreen}{(}}}{1}
           {\\)}{\textcolor{DarkOrange}{\textbackslash}{\textcolor{DarkGreen}{)}}}{1}
           {\\\\}{\textcolor{DarkOrange}{\textbackslash}{\textcolor{DarkGreen}{\textbackslash}}}{1}
           {\\.}{{\textcolor{DarkOrange}{\textbackslash}\textcolor{DarkGreen}{.}}}{1}
           {\\+}{{\textcolor{DarkOrange}{\textbackslash}\textcolor{DarkGreen}{+}}}{1}
           {\\-}{{\textcolor{DarkOrange}{\textbackslash}\textcolor{DarkGreen}{-}}}{1}
}
\lstset{
  postbreak=\raisebox{0ex}[0ex][0ex]{\ensuremath{\color{red}\hookrightarrow\space}},
  backgroundcolor=\color{LightGray!20!White},
  extendedchars=true,
  basicstyle=\small\ttfamily,
  showstringspaces=false,
  showspaces=false,
  numbers=left,
  numberstyle=\tiny,
  numbersep=4pt,
  tabsize=2,
  breaklines=true,
  showtabs=false,
  captionpos=b,
  frame=lines,
  columns=fixed,
  basewidth={.56em,.45em},
  fontadjust=true,
}

%---------------------------------
% listings inline
%---------------------------------
\newcommand{\myMIn}{\lstinline[style=MyMacroStyle,postbreak=]}
\newcommand{\myRIn}{\lstinline[language=MyRegex]}
\newcommand{\mySIn}{\lstinline[language=MyString]}
\newcommand{\myTIn}{\lstinline[]}

%---------------------------------
% Tikz Railroad styles
%---------------------------------
\definecolor{nonterminal_color}{RGB}{255, 230, 179}
\definecolor{terminal_color}{RGB}{230, 230, 230}
\tikzset{
  nonterminal/.style = {
    rectangle,
    rounded corners=3pt,
    minimum size=6mm,
    draw=nonterminal_color!80!black!95,
    fill=nonterminal_color,
    text height=1.5ex,
    text depth=.25ex,
    font=\ttfamily,
  },
  terminal/.style = {
    rounded rectangle,
    minimum size=6mm,
    draw=terminal_color!80!black!95,
    fill=terminal_color,
    text height=1.5ex,
    text depth=.25ex,
    font=\ttfamily,
  },
  StartEnd/.style = {
    circle,
    minimum size=3mm,
    draw=black!80,
    fill=black!60,
  },
  skip loop/.style = {
    to path={-- ++(0,#1) -| (\tikztotarget)}
  },
  hv path/.style = {
    to path={-| (\tikztotarget)}
  },
  vh path/.style = {
    to path={|- (\tikztotarget)}
  },
  railroad/.style = {
    >=stealth',
    black!50,
    text=black,
    thick,
    node distance = 4mm
  },
  graphs/railroad/.style = {
    edges=rounded corners,
    simple,
  },
  inner/.style={nonterminal, text height=, text depth=},
  outer/.style={terminal, rectangle, rounded corners=10pt, text height=, text depth=}
}

%---------------------------------
% An option to toggle inputs - used for tikz graphics
%---------------------------------
\newcommand{\myInput}[1]{%
  \adjustbox{max width=.9\linewidth}{%
    \scalebox{.9}{%
      \input{#1}%
    }%
  }%
}%
\newcommand{\myInputUnlimited}[1]{%
  \scalebox{.9}{%
    \input{#1}%
  }%
}%

%---------------------------------
% An Environment for code with captions
%---------------------------------
\DeclareCaptionType{myCodeEnvType}[List\-ing][List\-ing\-ver\-zei\-chnis]
\newenvironment{myCodeEnv}{
    \aftergroup\useignorespacesandallpars%
    \par%
    \vspace*{\baselineskip}%
    \captionsetup{type=myCodeEnvType}%
  }{%
    \par%
    \vspace*{\baselineskip}%
}

%---------------------------------
% Ignore Spaces
%---------------------------------
\def\useignorespacesandallpars#1\ignorespaces\fi{%
#1\fi\ignorespacesandallpars}

\makeatletter
\def\ignorespacesandallpars{%
  \@ifnextchar\par
    {\expandafter\ignorespacesandallpars\@gobble}%
    {}%
}
\makeatother


%---------------------------------
% Make the counter global for paracol
%---------------------------------
\globalcounter{myCodeEnvType}
\globalcounter{figure}
\globalcounter{section}
\globalcounter{footnote}

%---------------------------------
% Microtype optimisations
%---------------------------------
% http://www.khirevich.com/latex/microtype/
\SetProtrusion{encoding={*},family={bch},series={*},size={6,7}}
              {1={ ,750},2={ ,500},3={ ,500},4={ ,500},5={ ,500},
               6={ ,500},7={ ,600},8={ ,500},9={ ,500},0={ ,500}}
\SetExtraKerning[unit=space]
    {encoding={*}, family={bch}, series={*}, size={footnotesize,small,normalsize}}
    {\textendash={400,400}, % en-dash, add more space around it
     "28={ ,150}, % left bracket, add space from right
     "29={150, }, % right bracket, add space from left
     \textquotedblleft={ ,150}, % left quotation mark, space from right
     \textquotedblright={150, }} % right quotation mark, space from left
\SetExtraKerning[unit=space]
   {encoding={*}, family={qhv}, series={b}, size={large,Large}}
   {1={-200,-200},
    \textendash={400,400}}
\SetTracking{encoding={*}, shape=sc}{20}

%---------------------------------
% New page for new section
%---------------------------------
\addtokomafont{section}{\clearpage}

%---------------------------------
% Fixed width tabular
%---------------------------------
\newcolumntype{L}[1]{>{\raggedright\let\newline\\\arraybackslash\hspace{0pt}}m{#1}}
\newcolumntype{C}[1]{>{\centering\let\newline\\\arraybackslash\hspace{0pt}}m{#1}}
\newcolumntype{R}[1]{>{\raggedleft\let\newline\\\arraybackslash\hspace{0pt}}m{#1}}

%---------------------------------
% Tikz mark
%---------------------------------
\newcommand{\tikzmark}[2]{%
  \tikz[overlay,remember picture]%
  \node[text=black,yshift=.25\baselineskip] (#1) {#2};%
}

%---------------------------------
% Invisible non floating box for paracol to work with listings
%---------------------------------
\newtcolorbox{myInvBox}[1][]{
    width=\linewidth,
    boxsep=0cm,
    breakable,
    nobeforeafter,
    enhanced jigsaw,
    opacityfill=0,
    size=minimal,
    #1
}

%---------------------------------
% Savebox for one single special minted env
%---------------------------------
\newsavebox{\myMinSavBox}
