%---------------------------------
% Meta Variables
%---------------------------------
\newcommand{\MetaInstitute}{Hochschule Bremen}
\newcommand{\MetaUnit}{Bachelorarbeit}
\newcommand{\MetaTask}{Thesis}
\newcommand{\MetaTitle}{Konzeption und Implementierung einer Makrosprache in C++}
\newcommand{\MetaAuthorName}{Roland}
\newcommand{\MetaAuthorSurname}{J{\"a}ger}
\newcommand{\MetaStudentNumber}{360\,956}
\newcommand{\MetaAuthor}{\MetaAuthorName~\MetaAuthorSurname}

\documentclass[german,a4paper,12pt,parskip=half]{scrartcl}
\usepackage[T1]{fontenc} % utf8 <- produce real utf8 characters
\usepackage[utf8]{inputenc} % utf8 <- accept utf8 input characters
\usepackage{datetime}
\usepackage[german]{babel}
\usepackage[automark,headsepline]{scrlayer-scrpage}
\usepackage[vscale=0.75,vmarginratio={85:100},heightrounded]{geometry} % less margin at bottom
\usepackage[svgnames]{xcolor}
\usepackage{tablefootnote}
\usepackage{hyperref}
\usepackage[clean]{svg}
\usepackage{float}
\usepackage{minted}
\usepackage{graphicx} % for pdf and other graphics
\usepackage[
  backend=biber,
  sortlocale=de_DE,
  sortcites=true,
  url=false,
  doi=true,
  eprint=false
]{biblatex}
\usepackage{csquotes}
\usepackage{anyfontsize} % mute warnings?
\usepackage{microtype} % Subliminal refinements towards typographical perfection - eg. Hypernation
\usepackage{xspace} % set a space if not fullstop / end of sentence
\usepackage[all]{hypcap}
\usepackage{tikz}
\usepackage{tikzscale}
\usepackage{ifthen}
\usepackage{xstring}
\usepackage{calc}
\usepackage{pgfopts}
\usepackage[shellescape,latex]{gmp}

\usepackage[prependcaption,colorinlistoftodos,textwidth=5.5cm,textsize=footnotesize]{todonotes}

%---------------------------------
% Add extra width to the paper for todos
%---------------------------------
\makeatletter
  \if@todonotes@disabled
  \else
    \usepackage{background}

    % draw rule for real paper width
    \SetBgScale{1}
    \SetBgAngle{0}
    \SetBgColor{lightgray}
    \SetBgContents{\rule{.4pt}\paperheight}
    \SetBgHshift{9cm}

    % add more width to the paper
    \addtolength{\paperwidth}{3cm}
  \fi
\makeatother

%---------------------------------
% TikZ libraries
%---------------------------------
\usetikzlibrary{arrows}
\usetikzlibrary{shapes}
\usetikzlibrary{spy}
\usetikzlibrary{graphs}
\usetikzlibrary{calc}
\usetikzlibrary{positioning}

%---------------------------------
% TikZ scaling support
%---------------------------------
\let\OrgPgfTransformScale\pgftransformscale
\renewcommand*{\pgftransformscale}[1]{%
  \gdef\ScaleFactor{#1}%
  \OrgPgfTransformScale{#1}%
}
\def\ScaleFactor{1}

%---------------------------------
% footnotes in footnotes helper
%---------------------------------
\makeatletter
\newcommand{\spewnotes}{%
  \tfn@tablefootnoteprintout%
  \global\let\tfn@tablefootnoteprintout\relax%
  \gdef\tfn@fnt{0}%
}
\makeatother

%---------------------------------
% Shut up I know
%---------------------------------
\pdfsuppresswarningpagegroup=1

%---------------------------------
% Biographie
%---------------------------------
\addbibresource{lit.bib}

%---------------------------------
% SVG image configuration
%---------------------------------
\setsvg{inkscape = inkscape -z -C} % better svgs

%---------------------------------
% No Page break
%---------------------------------
\newenvironment{absolutelynopagebreak}
  {\par\nobreak\vfil\penalty20\vfilneg
   \vtop\bgroup}
  {\par\xdef\tpd{\the\prevdepth}\egroup
   \prevdepth=\tpd}

%---------------------------------
% hyphen used between two capitals
%---------------------------------
\newcommand{\capitalhyphen}{%
  \raisebox{0.24ex}{%
    \resizebox{0.4em}{%
      \height%
    }{-}%
  }%
  \kern-0.07em%
}

%---------------------------------
% Minted Code
%---------------------------------
\newcommand\myCMinco[5][cpp]{
  {%
    \ifthenelse{\equal{#2}{}}{%
    }{%
      \noindent\texttt{#2}\\[-2em]%
    }%
    ~\\[1em]% % BUG?
    \inputminted[
        frame=lines,
        framesep=2mm,
        bgcolor=LightGray!20!White, % % BUG?
        fontsize=\footnotesize,
        linenos,
        firstline=#4,
        lastline=#5,
      ]{#1}{#3}%
  }
}
\newcommand\myMinco[5][cpp]{
  {%
    \ifthenelse{\equal{#2}{}}{%
    }{%
      \noindent\texttt{#2}\\[-2em]%
    }%
    \inputminted[
        frame=lines,
        framesep=2mm,
        fontsize=\footnotesize,
        linenos,
        firstline=#4,
        lastline=#5,
      ]{#1}{#3}%
  }
}
\newcommand\myCMincoLine[5][cpp]{
  {%
    \ifthenelse{\equal{#2}{}}{%
    }{%
      \noindent\texttt{#2}\\[-2em]%
    }%
    ~\\[1em]% % BUG?
    \inputminted[
        frame=lines,
        framesep=2mm,
        bgcolor=LightGray!20!White, % % BUG?
        fontsize=\footnotesize,
        linenos,
        firstnumber=#4,
        firstline=#4,
        lastline=#5,
      ]{#1}{#3}%
  }
}
\newcommand\myMincoLine[5][cpp]{
  {%
    \ifthenelse{\equal{#2}{}}{%
    }{%
      \noindent\texttt{#2}\\[-2em]%
    }%
    \inputminted[
        frame=lines,
        framesep=2mm,
        fontsize=\footnotesize,
        linenos,
        firstnumber=#4,
        firstline=#4,
        lastline=#5,
      ]{#1}{#3}%
  }
}

%---------------------------------
% Minted inline
%---------------------------------
\newcommand{\myMinin}[1]{\mintinline{c++}{#1}}

%---------------------------------
% Horizontal line for title page
%---------------------------------
\newcommand{\HRule}{\rule{\linewidth}{0.2mm}}

%---------------------------------
% href Link as foot note
%---------------------------------
\newcommand\myFnurl[2]{%
  \ifthenelse{\equal{#1}{}}{%
    \footnote{\url{#2}}%
  }{%
    \href{#2}{#1}\footnote{\url{#2}}%
  }%
}

%---------------------------------
% Table in table for new line
%---------------------------------
\newcommand{\specialcell}[2][c]{% acceps t, b and c for vertival alignment
  \begin{tabular}[#1]{@{}l@{}}#2\end{tabular}}

%---------------------------------
% Meta data and Link Colour
%---------------------------------
\newcommand\myShade{70}

\definecolor{mylinkcolor}{RGB}{113, 31, 155}
\definecolor{mycitecolor}{RGB}{255, 189, 76}
\definecolor{myurlcolor}{RGB}{62, 106, 171}

\hypersetup{
  pdfauthor   = {\MetaAuthor},
  pdftitle    = {\MetaTitle},
  pdfsubject  = {\MetaUnit, \MetaTask},
  % pdfsubject  = {\MetaUnit},
  % pdfkeywords = {\MetaTitle, \MetaUnit, \MetaInstitute},
  pdfkeywords = {\MetaTitle, \MetaUnit, \MetaTask, \MetaInstitute},
  colorlinks  = true,
  linkcolor   = mylinkcolor!\myShade!black,
  citecolor   = mycitecolor!\myShade!black,
  urlcolor    = myurlcolor!\myShade!black,
}

%---------------------------------
% Minted color scheme
%---------------------------------
\usemintedstyle{borland}

%---------------------------------
% Fancy header
%---------------------------------
\clearpairofpagestyles
\lohead{\headmark}
\cofoot[\pagemark]{\pagemark}
\pagestyle{scrheadings}

%---------------------------------
% Start page count
%---------------------------------
\setcounter{page}{0}

\newcommand{\myTodo}[2][NOCAP]{
  \ifthenelse{\equal{#1}{NOCAP}}{%
    \todo[color=SandyBrown]{#2}%
  }{%
    \todo[color=SandyBrown, caption={#1}]{#2}%
  }%
}
\newcommand{\myFixme}[2][NOCAP]{
  \ifthenelse{\equal{#1}{NOCAP}}{%
    \todo[color=FireBrick!80]{#2}%
  }{%
    \todo[color=FireBrick!80, caption={#1}]{#2}%
  }%
}
\newcommand{\myQuestion}[2][NOCAP]{
  \ifthenelse{\equal{#1}{NOCAP}}{%
    \todo[color=LightBlue]{#2}%
  }{%
    \todo[color=LightBlue, caption={#1}]{#2}%
  }%
}

%---------------------------------
% Tikz Railroad styles
%---------------------------------
\definecolor{nonterminal_color}{RGB}{255, 213, 143}
\tikzset{
  nonterminal/.style = {
    rectangle,
    minimum size=6mm,
    very thick,
    draw=nonterminal_color!90!black!95,
    top color=white!20!nonterminal_color!30,
    bottom color=nonterminal_color,
    text height=1.5ex,
    text depth=.25ex,
    font=\ttfamily,
  },
  terminal/.style = {
    rounded rectangle,
    minimum size=6mm,
    very thick,
    draw=black!50,
    top color=white!30!black!15,
    bottom color=black!20,
    text height=1.5ex,
    text depth=.25ex,
    font=\ttfamily,
  },
  StartEnd/.style = {
    circle,
    minimum size=3mm,
    very thick,
    draw=black!80,
    top color=black!50,
    bottom color=black!80,
  },
  skip loop/.style = {
    to path={-- ++(0,#1) -| (\tikztotarget)}
  },
  hv path/.style = {
    to path={-| (\tikztotarget)}
  },
  vh path/.style = {
    to path={|- (\tikztotarget)}
  },
  railroad/.style = {
    >=stealth',
    black!50,
    text=black,
    thick,
    node distance = 4mm
  },
  graphs/railroad/.style = {
    edges=rounded corners,
    simple,
  }
}

